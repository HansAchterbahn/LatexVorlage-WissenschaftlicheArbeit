% !TeX spellcheck = de_DE

%  ******************************************************************************
%  * @file      tex/Einleitung                                                  *
%  * @author    Mario Hesse                                                     *
%  * @version   v0.1.0                                                          *
%  * @date      31.01.2019                                                      *
%  ******************************************************************************


\section{Einleitung}

Ein paar einleitende Worte. Wo schreibe ich die Arbeit? In welchem Kontext? Wer ist daran beteiligt?


	\subsection{Motivation}
	
	Aus welchem Grund wird dieses Thema bearbeitet? Wo ist der Vorteil? Welche wissenschaftliche Frage steht dahinter? Welches bisher nicht gelöste Problem wird gelöst?
	

	\subsection{Beschreibung der Ausgangssituation}
	
	Wie wird aktuell in dem Bereich gearbeitet? An welcher Stelle setzt die Arbeit an? Welche Vorarbeit wurde bisher geleistet?
	
	
	\subsection{Zielstellung} \label{sec:Zielstellung}
	
	Welcher Zustand soll erreicht werden? An welchem Punkt kann die Aufgabenstellung als gelöst angesehen werden?
	
	
	
	
	
	
	
	\subsection{Test für Glossaries}
	
	In diesem Abschnitt kann das Paket Glossaries getestet werden. Dazu müssen im Hauptdokument zwei Stellen einkommentiert werden. Diese Stellen befinden sich in Zeile 28 und 127 - 138.
	
	\paragraph{Abkürzungen}
	
	Wenn man die \gls{hawk} in einem Text erwähnt, muss man Sie beim ersten Mal ausschreiben. Bei der zweiten Erwähnung darf man die \gls{hawk} abkürzen. Glossaries macht das automatisch. Um das Glossar selbst zu erstellen, darf man nicht vergessen den Befehl \textit{Makeglossaries} auszuführen. In TeXstudio findet man ihn in der Menüleiste unter \textit{Tools/Befehle/Makeglossaries}. Dieser Befehl muss immer ausgeführt werden, wenn neue Glossar-Wörter in den Text kommen. 
	
	\paragraph{Glossar}
	
	Sollte ein Begriff wie \gls{Plasma} erklärungsbedürftig sein, kann man ihn in \textit{/tex/Glossaries.tex} definieren. Er landet im Glossar, sobald er mit \textit{$\backslash$gls\{KeyDesBegriffs\}} im Text erwähnt wird und die Befehle \textit{Makeglossaries} und \textit{Erstellen \& Anzeigen} ausgeführt wurden.
	
	Man kann auch Verlinkungen zwischen Abkürzungen und Glossar erstellen ein schönes Beispiel hierfür sieht man im Abkürzungsverzeichnis der \gls{sps}.
	
	
	\paragraph{Formelverzeichnis}
	
	Das Formelverzeichnis wird auf gleiche Weise angelegt, wie das Glossar und das Abkürzungsverzeichnis. Sobald die Werte eingeführt werden, kann man sie im Formelverzeichnis sehen. Um das zu demonstrieren seien hier \gls{symb:U}, \gls{symb:I} und \gls{symb:R} erwähnt. In \autoref{OhmschesGesetz} kann man sehen, dass die verlinkten Formelzeichen auch in eine Gleichung eingesetzt werden können.
	
	\begin{align}
		\text{\gls{symb:R}}=\frac{\text{\gls{symb:U}}}{\text{\gls{symb:I}}} \label{OhmschesGesetz}
		\end{align}
	
	
	


% !TeX spellcheck = de_DE

%  ******************************************************************************
%  * @file      tex/Hardware                                                    *
%  * @author    Mario Hesse                                                     *
%  * @version   v0.1.0                                                          *
%  * @date      31.01.2019                                                      *
%  ******************************************************************************

\section{Hardware} \label{sec:Hardware}

In diesem Abschnitt wird Hardware beschrieben, die im Zuge der Arbeit entsteht. Dazu könnten Tabellen wie \autoref{tab:DieErsteTabelle} nützlich sein.
		
		
\begin{table}[htbp]
	\centering																	% zentriert die Tabelle
	\caption{Die erste Tabelle}													% Tabellenbezeichnung (Taucht im Tabellenverzeichnis auf)
	\label{tab:DieErsteTabelle}													% Label mit dem auf die Tabelle verwiesen werden kann
	\begin{tabular}{l|c|r}														% Die Tabellen Spalten werden mit l,c oder r als linkbündig, zentriert oder rechtsbündig definiert; soviele Buchstaben wie geschrieben wurden, soviele Spalten existieren
		links angeordnet	& zentriert angeordnet	& rechts angeordnet \\ 		% Tabelleneinträge werden durch ein "&" getrennt und jede Reihe wird duch "\\" abgeschlossen
		\hline 																	% Erzeugt eine horizontale Linie
		links 				& zentriert 			& rechts \\					% Weitere Reihen schreibt man im gleichen Stil darunter
		l 					& z			 			& r \\					% Weitere Reihen schreibt man im gleichen Stil darunter
		
	\end{tabular}
\end{table}